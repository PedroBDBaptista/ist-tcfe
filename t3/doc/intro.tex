\section{Introduction}
\label{sec:introduction}

\par The objective of this laboratory assignment is to design and study a AC-DC converter. We will choose and implement the architecture of the converter in a way that maximizes the figure of merit, according to equation \ref{eq:merit}.

\begin{equation}
    M = \frac{1}{cost \times (ripple(v_0) + average (v_0 - 12) + 10^{-6})}
    \label{eq:merit}
\end{equation}

The circuit to be analysed is broadly described in figure \ref{fig:circuit}. It receives a AC input signal (230$V$/50$Hz$) and it's composed of a transformer with 10 to 1 spire ratio and galvanic isolation, followed by a Envelope Detector (ED) circuit and a Voltage Regulator (VR) circuit. The ideal output of the circuit is a 12$V$ DC signal. 

\begin{figure}[H]   
\centering
\includegraphics[width=10 cm]{geral.pdf}
\caption{AC-DC converter circuit scheme}
\label{fig:circuit}
\end{figure}

\par The ED and VR circuits are the ones whose architecture is up to our choosing. A more extensive description of these circuits will be presented in Section~\ref{sec:theoretical}, as well as a theoretical analysis of the implemented circuit, with the help of \textit{Octave} software. In Section~\ref{sec:simulation}, the circuit is analysed by simulation, using the software \textit{ngspice}, and the results are compared to the theoretical results obtained in Section~\ref{sec:theoretical}. The conclusions of this study are outlined in Section~\ref{sec:conclusion}.