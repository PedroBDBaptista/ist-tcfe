\section{Conclusion}
\label{sec:conclusion}

Comparing the results obtained through our theoretical analysis (with the help of Octave
software) with the values given to us by emph{NGSpice}'s simulation, and, furthermore, by checking the
error tables \ref{error_tensoes} and \ref{error_current}, we can claim that both sets match. The percentual
errors are very low (far less than 1\%) and so we conclude the Mesh and Node Analyses were both done
correctly and can be used separately to find currents or voltages to describe the circuit.

\par

This consistency follows because \emph{Ngspice} works by solving a set of simultaneous equations
based on the Kirchhof's Laws. Thus, the simulator operates by solving a matrix of linear equations much
like the one which the theoretical analysis was based on. The errors do exist because \emph{Ngspice}'s output
precision is set by default to 6 digits (one can increase the precision of the calculations up to, approximately,
16 digits) and \emph{Octave}'s output' gives us more precise results.
