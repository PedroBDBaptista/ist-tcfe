\section{Simulation Analysis}
\label{sec:simulation}

\subsection{Operating Point Analysis}

Table~\ref{tab:op} shows the simulated operating point results for the circuit
under analysis. Compared to the theoretical analysis results, one notices the
following differences: describe and explain the differences.

\begin{table}[H]
  \centering
  \begin{tabular}{|l|r|}
    \hline
    {\bf Name} & {\bf Value [A or V]} \\ \hline
    @cb[i] & 0.000000e+00\\ \hline
@ce[i] & 0.000000e+00\\ \hline
@q1[ib] & 7.022567e-05\\ \hline
@q1[ic] & 1.404513e-02\\ \hline
@q1[ie] & -1.41154e-02\\ \hline
@q1[is] & 5.765392e-12\\ \hline
@rc[i] & 1.411536e-02\\ \hline
@re[i] & 1.411536e-02\\ \hline
@rf[i] & 7.022567e-05\\ \hline
@rs[i] & 0.000000e+00\\ \hline
v(1) & 0.000000e+00\\ \hline
v(2) & 0.000000e+00\\ \hline
base & 2.254108e+00\\ \hline
coll & 5.765392e+00\\ \hline
emit & 1.411536e+00\\ \hline
vcc & 1.000000e+01\\ \hline

  \end{tabular}
  \caption{Operating point. A variable preceded by @ is of type {\em current}
    and expressed in Ampere; other variables are of type {\it voltage} and expressed in
    Volt.}
  \label{tab:op}
\end{table}

[EXPLICAR SINAIS TROCADOS NAS CORRENTES DE VA E VC DEVIDO A CONVENÇÃO DO NGSPICE (CONTRÁRIA À ASSUMIDA POR NÓS)]
[EXPLICAR V8 COMO FONTE FICTICIA PARA CALCULO DO NGSPICE]

ERROS PERCENTUAIS DAS CORRENTES E TENSOES. CONSIDERANDO VALOR REAL O OBTIDO PELO OCTAVE.
FORMULA = ABS(X(NGSIPE)-X(OCTAVE) ) / X(OCTAVE) *100


%formula do erro
\begin{equation}
  \centering
  \epsilon (x_e)=\frac{|x_e-x_r|}{x_r} \times 100\,\%
  \label{eq:error}
\end{equation}

We have considered the value obtained by the mashes/nodes methods as the "real" one ($x_r$) and the value obtained through \emph{ngspice} as the experimental one ($x_e$). It is important to notice, once more, that the values from \emph{ngspice} for the current going through the voltages souces are the simetric of the ones we have considered. The percentual error computation had this in mind.

ERROS CORRENTES

\begin{table}[H]
  \centering
  \begin{tabular}{|l|r|}
    \hline
    {\bf Name} & {\bf Value in \%} \\ \hline
    \input{../mat/error_tensoes}
  \end{tabular}
  \caption{Currents' percentual errors}
  \label{tab:op}
\end{table}

ERROS TENSOES

\begin{table}[H]
  \centering
  \begin{tabular}{|l|r|}
    \hline
    {\bf Name} & {\bf Value in \%} \\ \hline
    \input{../mat/error_current}
  \end{tabular}
  \caption{Voltages' percentual errors }
  \label{tab:op}
\end{table}
