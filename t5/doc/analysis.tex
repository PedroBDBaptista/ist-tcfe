\section{Theoretical Analysis}
\label{sec:theoretical}



\subsection{Gain Stage}
\par The main purpose of this part of the circuit is, as the name suggests, to increase the voltage of the output signal, producing, therefore, a high Voltage Gain.\\
The figure of the circuit is presented in figure \ref{fig:gain_stage}


We start by giving a simple explanation as to the existance of some of the presented components. For example, $C_s$ is a coupling capacitor whose job is to eliminate the DC component from the voltage source $v_s$. The bias circuit (represented in red in figure \ref{fig:gain_stage}) aims to guarantee that the voltage at the source is such that allows the transistor to conduct normally, making sure it is in the forward active region.
Resistor $R_E$ is used to stabilize the temperature effects on the circuit, however, it lowers the gain. For that reason, capacitor $C_E$ is placed in parallel to it. With this configuration we can see that for low frequencies (DC) the capacitor works as an open circuit, for medium/high frequencies the capacitor works as a short circuit, bypassing $R_E$, thus increasing the gain again.\\

We are now interested in doing an operating point analysis of the circuit. To do so, we start by computing the Thevenin's equivalent of the bias circuit. At an operating point analysis, all capacitors behave as open circuits. We now present the circuit obtained with the Thevenin's equivalent.


% \begin{figure}[H]
% \centering
% \includegraphics[width=0.5\linewidth]{gainstage_equiv.pdf}
% \caption{Gain Stage Circuit Thevenin's equivalent}
% \label{fig:gain_stage_circuit}
% \end{figure}




\begin{equation}
GAIN=\Bigg( \frac{ \frac{1}{j\omega C_2}}{\frac{1}{j\omega C_2}+R_2} \Bigg) \Bigg( \frac{R_1}{R_1+\frac{1}{j\omega C_1}}   \Bigg)  \Bigg( 1+\frac{R_4}{R_3}  \Bigg)
\end{equation}



\begin{center}
    \begin{table}[H]
        \centering
        \begin{tabular}{c|c|c}
          \textbf{Quantity} & \textbf{Theoretical}  & \textbf{Simulation}  \\
          \hline
            Lower Cut Frequency (Hz) &  ---- &  15.532 \\
            Upper Cut Frequency (MHz)&  ---- &  2.5446 \\
        \end{tabular}
        \caption{\textit{Octave}'s results vs. \textit{ngspice}'s results}
    \end{table}
\end{center}

