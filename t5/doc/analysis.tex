\section{Theoretical Analysis}
\label{sec:theoretical}

In this section we are going to theoretically analyze the circuit, deriving results with the help of the \textit{Octave} software and compare them with the simulation's results.
The circuit to be analyzed is presented below:

\begin{figure}[H]
\centering
  \includegraphics[width=.5\linewidth]{geral_2.pdf}
  \caption{Active Non-inverting OP AMP Bandpass Filter}
  \label{fig:geral2}
\end{figure}

As you can see, the circuit is divided in 3 stages: the High Pass Filter Stage, the Amplifier Stage and the Low Pass Filter Stage, which serve the function of, respectively, block out low frequencies, amplify the voltage signal and block out high frequencies.
Furthermore, the OP AMP is implemented in the circuit in a non-inverting configuration. This way, the circuit acts as an Active Non-inverting OP AMP Bandpass Filter.\par
Now we proceed to compute the input and output impedances, as well as the gain of the circuit.

\subsection{Input and Output Impedances}

The input impedance is computed as:
\begin{equation}
Z_{I}=\left.\frac{V_{i}}{I_{i}}\right\vert_{Z_L=\infty}
\end{equation}
where $Z_{L}$ is the load impedance and $I_i$ is the current that passes through the voltage source $V_i$. Since no current enters an OP AMP (in the $+$ and $-$ terminals), this leaves us with a capacitor in series with a resistor.
Thus, the input impedance is:

\begin{equation}
Z_{I}=\frac{1}{jwC_1}+R_1
\end{equation}

The output impedance is computed as:

\begin{equation}
Z_{O}=\left.\frac{V_{o}}{I_{o}}\right\vert_{V_i=0}
\end{equation}
Again, since no current enters an OP AMP and $V_i=0$, then $V_{+}=V_{-}=0$  and we are left with a capacitor $C_2$, which is a series of 2 capacitors $C_{21}$ and $C_{22}$, in parallel with a series of resistors $R_2$, $Z_{OPAMP}=0$.
This way, the output impedance is:

\begin{equation}
Z_{O}=\frac{1}{\frac{1}{R_2}+jwC_2}
\end{equation}

where

\begin{equation}
C_2=\frac{1}{jw(C_{21}+C_{22})}
\end{equation}

\subsection{Gain}

To compute the gain, we can divide the problem in 3 sections. Starting with the High Pass Filter Stage and using a Voltage Divider, we get:

\begin{equation}
V_{+}=\frac{R_1}{R_1+1/(jwC_1)}V_i
\label{Eq:V+}
\end{equation}

With this expression we can already see the effect of this stage. As the name suggests, for high frequencies $V_{+}$ $\to$ $V_i$ and for low frequencies $V_{+}$ $\to$ $0$.
Thus, the High Pass Filter Stage blocks the signals with low frequencies.

Now, going through the amplifier stage and by inspection we see that $V_a=V_{-}+R_4I_4$, where $I_4=I_3=(V_{-}-0)/R_3$. Thus, we get:

\begin{equation}
V_a=V_{-}+R_4\frac{V_{-}}{R_3}  \Leftrightarrow V_a = V_{-}\left(1+\frac{R_4}{R_3}\right)
\label{Eq:Va}
\end{equation}

This equation reveals the amplifying properties of this stage. The greater $R_4$ is compared to $R_3$, the greater $V_a$ will be. This is why this stage is called the amplifier stage.\par

Last but not least, we get to the Low Pass Filter Stage. Using a Voltage Divider:

\begin{equation}
  V_o=\frac{\frac{1}{jwC_2}}{\frac{1}{jwC_2}+R_2}V_a
\label{Eq:Vo}
\end{equation}

This explains why this is called the Low Pass Filter Stage. For low frequencies $V_o$ $\to$ $V_a$, but for high frequencies $V_o$ $\to$ $0$.\par

Now, using equations (\ref{Eq:V+}) , (\ref{Eq:Va}) and (\ref{Eq:Vo}) and remembering that $V_{+}=V_{-}$ thanks to negative feedback, the gain is given by:
\begin{equation}
\frac{V_o(\omega)}{V_i(\omega)}=\Bigg( \frac{ \frac{1}{j\omega C_2}}{\frac{1}{j\omega C_2}+R_2} \Bigg) \Bigg( \frac{R_1}{R_1+\frac{1}{j\omega C_1}}   \Bigg)  \Bigg( 1+\frac{R_4}{R_3}  \Bigg)
\end{equation}

\subsection{Theoretical results and comparison}

With the help of \textit{Octave} we plotted the gain as a function of the frequency $f=w/(2\pi)$. The following figure shows the frequency response from the theoretical analysis compared with the one we got from the simulation:
\begin{figure}[H]
\centering
\begin{subfigure}{.5\textwidth}
  \centering
  \includegraphics[width=.8\linewidth]{ganho.eps}
  \caption{\textit{Octave}'s gain frequency response}
  \label{fig:Ngspicegain(f)}
\end{subfigure}%
\begin{subfigure}{.5\textwidth}
  \centering
  \includegraphics[width=.8\linewidth]{../sim/vo1db.pdf}
  \caption{\textit{Ngspice}'s gain frequency response}
  \label{fig:Ngspicegain(f)}
\end{subfigure}
\caption{Side-by-side comparison of \textit{Ngspice}'s and \textit{Octave}'s frequency response}
\label{fig:Gaincomparison}
\end{figure}

We also present a table with key results alongside their simulation counterparts and their respective errors assuming ngspice's results as the exact results:

\begin{center}
    \begin{table}[H]
        \centering
        \begin{tabular}{c|c|c|c}
          \textbf{Quantity} & \textbf{Theoretical}  & \textbf{Simulation} & \textbf{Error}\\
          \hline

            Lower Cut Frequency (Hz) &  406.78 &  406.81 & 0.007$\%$ \\
            Upper Cut Frequency (Hz)&  2576.56 &  2481.3 & 3.8$\%$ \\
            Central Frequency (Hz)& 1023.77 &  1004.69 & 1.91$\%$  \\
            Gain (dB) &  36.564  & 36.532 & 0.09$\%$ \\
            Input Impedance ($\Omega$) & 1000.00-723.43i & 999.99-723.53i & 0.001$\%$ and 0.01$\%$\\
            Output Impedance ($\Omega$) & 676.73-467.72i & 680.05-466.90i & 1.03$\%$ and 103$\%$\\
        \end{tabular}
        \caption{\textit{Octave}'s results vs. \textit{ngspice}'s results}
    \end{table}
\end{center}
