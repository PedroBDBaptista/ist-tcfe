\section{Conclusion}
\label{sec:conclusion}

Comparing the results obtained in the theoretical analysis  with the ones acquired from \textit{ngspice} and
also checking the \textit{Octave} plots with the simulation's plots, we can say that they are essentially the same. We can thus conclude
that our theoretical analysis matches the simulated circuit and can effectively describe the system's voltages and currents in stationary regimes,
their time evolution and their frequency dependence, proving that it represents a really good approximation of reality.

\par

The operating point analyses match because \textit{ngspice} works by solving a matrix of linear equations based on the Kirchhoff's Laws, much like
what we do on the theoretical analysis. The transient analyses also match because the differential equations that rule the system's behaviours have
explicit solutions, and we can easily apply these solution, having only to take precautions with the initial conditions. \textit{Ngspice} solves these
differential equations numerically after we declare the initial conditions. Naturally, since the initial conditions , found in the operating point analysis, match,
the transient analyses must match as well. In the frequency analysis, \textit{ngspice} simulates the steady-state solution of the system for multiple frequencies.
This is what we essentially do in the theoretical analysis, exploiting the phasor notation for each frequency, which assumes that steady-state solution for the component's currents and node
voltages all oscillate with the same frequency as the voltage source.
