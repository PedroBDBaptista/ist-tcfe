\section{Simulation Analysis}
\label{sec:simulation}

\par We now proceed to perform a computational simulation of the amplifier, using the \textit{Ngspice} software. The implemented gain stage circuit is represented in figure [\ref{fig:gain_stage}].

\begin{figure}[H]
\centering
  \includegraphics[width=.5\linewidth]{gainstage.pdf}
  \caption{Gain stage circuit}
  \label{fig:gain_stage}
\end{figure}

\par This circuit is composed of the following components: four resistors ($R_{B1}=80k\Omega$, $R_{B2}=20k\Omega$, $R_{C}=1k\Omega$ and $R_{E1}=100\Omega$), a coupling capacitor ($C_s = 1 mF$), a bypass capacitor ($C_b=2.74mF$) a NPN transistor ($Q_1$, model BC547A) and a constant voltage source ($V_{cc}=12V$). Identified with red, the bias circuit ensures the adequate voltage at $Q_1$'s base terminal.

\begin{figure}[H]
\centering
  \includegraphics[width=.5\linewidth]{output_stage.pdf}
  \caption{Output Stage circuit}
  \label{fig:output_stage}
\end{figure}

\par The second stage, shown in figure \ref{fig:output_stage}, consists of a PNP transistor ($Q_2$, model BC557A), a resistor ($R_{E2}=5\Omega$) and a coupling capacitor ($C_2=3mF$). It's purpose is to minimize the output impedance, so that the load ($R_L=8\Omega$) can be safely conected and the signal isn't significantly damped.

\begin{figure}[H]
\centering
  \includegraphics[width=.5\linewidth]{geral_completo.pdf}
  \caption{Audio Amplifier circuit, detailed scheme}
  \label{fig:complete_circuit}
\end{figure}

\par The values of the components were optimized to satisfy the operation of $Q_1$ and $Q_2$ in the foward-active region (FAR), the minimization of the lower cutoff frequency, the maximization of the band width, as the adequate input and output impedances.

\par We now show the gain of the entire circuit $v_o / v_s$ as a function of frequency, from 10$Hz$ to 100$MHz$, according to the \textit{Ngspice} simulation. This is obtained through the voltage at node \textit{out}, using the fact that the \textit{ac} analysis of the software automatically uses an incremental version of the circuit, allowing the use of the source voltage amplitude as $v_s = 1V$, withot further implications.

\vspace{-2cm}
\begin{figure}[H]
\centering
  \includegraphics[width=.8\linewidth]{../sim/vo2_db.pdf}
  \caption{Gain's frequency response of the amplifier circuit}
  \label{fig:gain_stage_gain(freq)}
\end{figure}

\par For these conditions, the band gain is $v_o/v_s = 26.142 dB$. The lower and upper cutoff frequencies are, respectively and aproximately, $15.532 Hz$ and $2.5446 MHz$. Thus, the bandwidth is $2.5445 MHz$. The input impedance is $Z_i=606.40\Omega$. The output impedance is $Z_o=4.1295 \Omega$. These values have been taken from the \textit{log} files in the \textit{t4/sim} folder. The figure of merit obtained with these parameters is $M=624,5$. These results are further discussed in the next section.
