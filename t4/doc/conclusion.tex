\section{Conclusion}
\label{sec:conclusion}

Comparing the results from the simulation with the ones gotten from the theoretical analysis, we can conclude that they show several discrepancies.

One thing we noticed is that the relations derived from the theoretical analysis don't always hold. For example, the expression for the final output impedance states that the lower the resistance RE2, the lower the impedance, yet many times lowering this value on \textit{ngspice} would not lead
to a decrease in this impedance. On the contrary, the impedance grew larger. This and other discrepancies show that the model we used to describe the system in the theoretical analysis doesn't exactly describe what happens in reality and is rather incomplete. This assumption is justified by comparing
the simple transistor models used in the theoretical analysis with the simulator's transistor models, which are much more complex, having a large number of parameters to take into account and also introducing parasitic capacitors and inductors that better help describe the transistor's behaviour.
This explains the large errors we got. We also noticed that small errors in the Operating Point Analysis propagated to larger errors in the incremental analysis.
\par
The Amplifier we built is not ideal, even though the gain is acceptable, because the impedances are not really compatible, namely the input impedance with resistor $R_s$ and, most importantly, the output impedance with the load resistor $R_L$. We tried to decrease this last impedance, lowering
$R_{E2}$ to a value of $1\Omega$. This showed to be great for $Zo$, but it would also lower the output stage's input impedance, which would make the whole output stage incompatible with the gain stage. ZO1 would become larger than ZI2 ($Z01>>ZI2$) which is the opposite of what we want. This would take a toll on the
amplifier's gain, since the gain stage's output signal would get damped in the output stage's terminals. \par

Nevertheless, the total cost of the circuit we built is 6857.4 monetary units, obtaining a merit of $M=624.5$.
