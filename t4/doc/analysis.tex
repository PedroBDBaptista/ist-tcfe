\section{Theoretical Analysis}
\label{sec:theoretical}



\subsection{Gain Stage}
\par The main purpose of this part of the circuit is, as the name suggests, to increase the voltage of the output signal, producing, therefore, a high Voltage Gain.\\
We now present a more detailed of the architecture of the gain stage.

\begin{figure}[h!]
\centering
\includegraphics[width=0.5\linewidth]{gainstage.pdf}
\caption{Gain Stage Circuit}
\label{fig:gain_stage_circuit}
\end{figure}



We start by giving a simple explanation as to the existance of some of the presented components. For example, $C_s$ is a coupling capacitor whose job is to eliminate the DC component from the voltage source $v_s$. The bias circuit (represented in red in figure \ref{fig:gain_stage_circuit}) aims to guarantee that the voltage at the source is such that allows the transistor to conduct normally, making sure it is in the forward active region.
Resistor $R_E$ is used to stabilize the temperature effects on the circuit, however, it lowers the gain. For that reason, capacitor $C_E$ is placed in parallel to it. With this configuration we can see that for low frequencies (DC) the capacitor works as an open circuit, for medium/high frequencies the capacitor works as a short circuit, bypassing $R_E$, thus increasing the gain again.\\

We are now interested in doing an operating point analysis of the circuit. To do so, we start by computing the Thevenin's equivalent of the bias circuit. At an operating point analysis, all capacitors behave as open circuits. We now present the circuit obtained with the Thevenin's equivalent.


\begin{figure}[H]
\centering
\includegraphics[width=0.5\linewidth]{gainstage_equiv.pdf}
\caption{Gain Stage Circuit}
\label{fig:gain_stage_circuit}
\end{figure}


In this stage, is particularly important that we calculate the voltage $V_{CE}$ and make sure that it is greater than $V_{BEON}$ confirming that the transistor is working in the forward active region (FAR).\\

The values for other important quantities will be presented alongside the ones obtained in the simulation in order to ease its analysis.\\

We shall now focus on finding a way to calculate the expected voltage gain of the circuit. We do so by considering an incremental circuit for medium frequencies. Remembering that for medium frequencies the capacitor $C_E$ behaves like a short circuit which is equivalent to consider $R_E=0$.

\begin{figure}[H]
\centering
\includegraphics[width=0.5\linewidth]{gainstage_increm.pdf}
\caption{Gain Stage Incremental Circuit}
\label{fig:gain_stage_incremental_circuit}
\end{figure}

Using mesh analysis one can derive the following expressions for the voltage gain of the circuit.

\begin{equation}
v_o=-g_m(R_C||r_o)v_{\pi}
\end{equation}

\begin{equation}
v_{\pi}=\frac{r_{\pi}||R_{B1}||R_{B2}}{R_s+r_{\pi}||R_{B1}||R_{B2}}
\end{equation}

\begin{equation}
gain=\frac{v_o}{v_s}=-g_m(R_C||r_o)\frac{r_{\pi}||R_{B1}||R_{B2}}{R_s+r_{\pi}||R_{B1}||R_{B2}}
\end{equation}

One can also obtain expressions for the input and output impegances of the circuit gain. Those are as follows:


\begin{equation}
Z_I=R_{B1}||R_{B2}||r_{\pi}
\end{equation}

\begin{equation}
Z_o=R_C||r_o
\end{equation}

\subsection{Output Stage}
The impedances obtained from the Gain stage are not compatible neither with the input's resistance $R_s$ nor with the resistance from the speaker $R_{speaker}=8\Omega$. In order to maintain the integrity of the signal, one must aim for a large value for $Z_I$ to not degrade the input signal and a low value for $Z_o$ to not degrade the output signal.\\

We start by introducing the circuit with which we are going to work. The figure below shows the drawn circuit.

\begin{figure}[h!]
\centering
\includegraphics[width=0.5\linewidth]{output_stage.pdf}
\caption{Output Stage Circuit}
\label{fig:output_stage_figure}
\end{figure}

Much like in the previous section, we are interested in doing an operation point analysis. In such, the capacitor will behave as an open circuit and the figure \ref{fig:output stage figure} can be simplified to:

\begin{figure}[h!]
\centering
\includegraphics[width=0.5\linewidth]{output_stage_OP.pdf}
\caption{Output Stage Circuit for OP Analysis}
\label{fig:output_stage_figure_OP}
\end{figure}


Computing the voltage gain vou this section of the circuit, we end up with:

\begin{equation}
gain=\frac{v_o}{v_i}=\frac{g_m}{g_m+g{\pi}+g_E+g_o} \approx 1
\end{equation}

The impedances are given by:

\begin{equation}
Z_I=\frac{g_m+g{\pi}+g_E+g_o}{g{\pi}(g{\pi}+g_E+g_o)}
\end{equation}


\begin{equation}
Z_o=\frac{1}{g_m+g{\pi}+g_E+g_o}
\end{equation}

\subsection {Theoretical results and comparison}

We now proceed to list some of the theoretical results and comparing them to the ones obtained in the simulation:

\begin{center}
    \begin{table}[H]
        \centering
        \begin{tabular}{c|c|c}
          \textbf{Quantity} & \textbf{Theoretical}  & \textbf{Simulation}  \\
          \hline
            $Gain_{gain_stage}$ (dB) &	48.392  & ---- \\
            $Gain_{output_stage}$  &	0.99195  & ----\\
            $Gain_{total}$ (dB)   &	42.585 & 26.142 \\
            ZI1($\Omega$)  &	484.43 & ----\\
            Z01($\Omega$)  &	886.28 & ----\\
            ZI2($\Omega$)  &	429.94 & ----\\
            Z02($\Omega$) & 0.015109 & ----\\
            Zi ($\Omega$)& 484.43 & 606.40\\
            Zo ($\Omega$)& 1.9972 & 4.1295\\
        \end{tabular}
        \caption{\textit{Octave}'s results vs. \textit{ngspice}'s results}
    \end{table}
\end{center}

First we should take into account that the theoretical analysis was done assuming we were in a medium frequency regime in which
the capacitor works as a short-circuit.
This means that the $Gain$ values are the bandwidth gain values (we did not perform a frequency analysis in the theoretical analysis,
we just assumed we were working on the bandwidth frequencies).
$Gain_{gain_stage}$ is high and $Gain_{output_stage} \approx 1$ corroborating the correct performance of an Amplifier. The total gain, $Gain_{total}$, is still high
but does not match the \textit{ngspice}'s gain, presenting an error of $E(\%)=62.9\% $. The decrease from $Gain_{gain_stage}$ to $Gain_{total}$ comes from (not only!) the impedances,
which we will comment below:\par

The gain stage's input impedance, $ZI1$, is $484.43\Omega$, which is not a great input impedance when compared to $R_s=100\Omega$. This means that the signal from $V_s$ is
attenuated in the gain stage's terminals. Ideally, we would want $ZI1>>R_s$. The output stage's output impedance, $Z02=0.015109 \Omega$, is very low compared to the Load resistance $R_L=8\Omega$.
Even though this is not the final output impedance, it showcases one ot the output stage's characteristcs : low output impedance, which is crucial for the signal to not be attenuated in the load resistance.
Now $Z01$ and $ZI2$ are not ideal. We would prefer $ZI2$ to be way larger than $Z01$ so that the signal between the gain stage's terminals would not be damped when transitioning to the output stage's terminals.
Effectively, this is what happens, meaning it's one of the reasons $Gain_{total}$ decreases when compared to $Gain_{gain_stage}$.\par
Zi=ZI1 and comparing it to \textit{ngspice}'s input impedance we get an error of $E(\%)=20.1\%$. The final output impedance is $Zo=1.9972\Omega$ which is not ideal when comparing it to the load resistance. Ideally,
we would want $Zo<<R_L$. Comparing it to \textit{ngspice}'s output impedance, we get an error of $E(\%)=51.6\%$.
\par
We would like to discuss the role of $R_c$ in the $Gain_{gain_stage}$ of the circuit. The greater its value, the greater the gain will be yet when one increases $R_c$ above a certain level, the transistor no longer works in
the forward active region.
